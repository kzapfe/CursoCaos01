% --------------------------------------------------------------
% This is all preamble stuff that you don't have to worry about.
% Head down to where it says "Start here"
% --------------------------------------------------------------
 
\documentclass[12pt]{article}
 
\usepackage[margin=1in]{geometry} 
\usepackage{amsmath,amsthm,amssymb}
\usepackage[utf8]{inputenc}
\usepackage[spanish]{babel}
\usepackage{graphicx} 

\newcommand{\N}{\mathbb{N}}
\newcommand{\Z}{\mathbb{Z}}
\newcommand{\M}{\mathbb{M}}
\newcommand{\rd}{\!\!\mathrm{d}}
 
\newenvironment{theorem}[2][Teorema]{\begin{trivlist}
\item[\hskip \labelsep {\bfseries #1}\hskip \labelsep {\bfseries #2.}]}{\end{trivlist}}
\newenvironment{lemma}[2][Lema]{\begin{trivlist}
\item[\hskip \labelsep {\bfseries #1}\hskip \labelsep {\bfseries #2.}]}{\end{trivlist}}
\newenvironment{exercise}[2][Ejercicio]{\begin{trivlist}
\item[\hskip \labelsep {\bfseries #1}\hskip \labelsep {\bfseries #2.}]}{\end{trivlist}}
\newenvironment{problem}[2][Problema]{\begin{trivlist}
\item[\hskip \labelsep {\bfseries #1}\hskip \labelsep {\bfseries #2.}]}{\end{trivlist}}
\newenvironment{question}[2][Pregunta]{\begin{trivlist}
\item[\hskip \labelsep {\bfseries #1}\hskip \labelsep {\bfseries #2.}]}{\end{trivlist}}
\newenvironment{corollary}[2][Corolario]{\begin{trivlist}
\item[\hskip \labelsep {\bfseries #1}\hskip \labelsep {\bfseries #2.}]}{\end{trivlist}}
 
\begin{document}
 
% --------------------------------------------------------------
%                         Start here
% --------------------------------------------------------------
 
\title{Tarea Semanal 2}%replace X with the appropriate number
\author{Karel Zapfe\\ %replace with your name
Introducción al Caos y Dinámica No Lineal} %if necessary, replace with your course title
 
\maketitle
 Supongamos que tenemos un sistema dinámico en un espacio fase $\M$ definido 
por la transformación $T$ (no necesariamente invertible) y que preserva
una medida $\mu$. Por simplicidad supondremos que el sistema es discreto
en el tiempo (las iteraciones de $T$ están indexadas por un número entero).
La medida total de nuesto espacio es 1, es decir, $\mu(\M)=1$.
La noción que usan los matematicos de  \emph{``casi donde sea''} 
quiere decir, en todos lados excepto en conjuntos de medica cero. 
 

\begin{lemma}{Definiciones Equivalentes} 
  Las siguientes nociones de Ergodicidad son equivalentes. 
  \begin{itemize}
  \item Los únicos conjuntos (medibles) que se mantienen invariantes bajo la transformación son aquellos de medida 0 o 1, excepto subconjuntos despreciables:
    \begin{equation*}
      T^{-1}(A)=A \quad ( \mod 0 )\Leftrightarrow \mu(A)=0, 1
    \end{equation*}
  \item El promedio de cualquier función integrable a lo largo de una órbita
    coincide con el promedio del espacio fase, para casi cualquier condición
    inicial:
    \begin{equation*}
      \lim_{N\rightarrow \infty} \frac{1}{N}\sum_{t=0}^{N} f(T^t(x)) = \int_\M \rd x f(x)
    \end{equation*}
  \item Las únicas funciones invariantes frente a la transformación 
    son constantes \emph{casi donde sea}:
    \begin{equation*}
      f(T(x))=f(x) \Leftrightarrow f(x)=\text{constante}.
    \end{equation*}
  \end{itemize}

\end{lemma}

Para entender bien estas nociones deberán saber que no todos los conjuntos
son \emph{medibles}. Existen subconjuntos de un espacio que donde la 
nocion de medida no esta definida. Vamos a hacer un poco de investigación
de ``campo'' al respecto.


\begin{question}{1}
Averiguen con un amigo matemático que haya cursado Análisis I y II o con algún
físico que haya llevado un curso de Matemáticas avanzadas que
quiere decir $\sigma$-álgebra (sigma-álgebra) en el contexto de 
subconjuntos de un conjunto. 
De paso averiguen que quiere decir álgebra en el mismo contexto y que son
los borelianos. Escribanme las definiciones de forma que cualquiera de 
ustedes las entienda, pero que sean rigurosas. Las medidas están definidas
para conjuntos que pertenecen esta $\sigma$-álgebra. Nosotros
usaremos medidas suaves, como longitud, área, volumen, sobre los borelianos.
\end{question}


\begin{question}{2}
Para el mapeo logístico $T(x)=kx(1-x)$, encontrar analíticamente
las siguientes cualidades. 
\begin{itemize} 
\item Los puntos fijos (puntos de periodo 1) en función de $k$.
\item La derivada del mapeo evaluada en el mismo punto. ¿En que valor
de $k$ deja de ser estable? ¿que otros valores \emph{chistosos} ocurren?
\item Hagan unas gráficas bonitas que ilustren los dos puntos anteriores. 
\item Punto Extra: ¿Pueden hacer lo mismo para los puntos de periodo 2?
  ¿En que momento aparecen, al menos?
\end{itemize}
\end{question}
 
\begin{question}{3}
Lean el \emph{paper} adjunto, \emph{``Chaos in a double Pendulum''}.
\end{question}

\end{document}