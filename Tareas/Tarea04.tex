% --------------------------------------------------------------
% This is all preamble stuff that you don't have to worry about.
% Head down to where it says "Start here"
% --------------------------------------------------------------
 
\documentclass[12pt]{article}
 
\usepackage[margin=1in]{geometry} 
\usepackage{amsmath,amsthm,amssymb}
\usepackage[utf8]{inputenc}
\usepackage[spanish]{babel}
\usepackage{graphicx} 
\usepackage{float} %Forzamos a las figuras a estar DONDE QUEREMOS
\usepackage{url}

\newcommand{\N}{\mathbb{N}}
\newcommand{\Z}{\mathbb{Z}}
\newcommand{\M}{\mathbb{M}}
\newcommand{\rd}{\!\!\mathrm{d}}
 
\newenvironment{theorem}[2][Teorema]{\begin{trivlist}
\item[\hskip \labelsep {\bfseries #1}\hskip \labelsep {\bfseries #2.}]}{\end{trivlist}}
\newenvironment{lemma}[2][Lema]{\begin{trivlist}
\item[\hskip \labelsep {\bfseries #1}\hskip \labelsep {\bfseries #2.}]}{\end{trivlist}}
\newenvironment{exercise}[2][Ejercicio]{\begin{trivlist}
\item[\hskip \labelsep {\bfseries #1}\hskip \labelsep {\bfseries #2.}]}{\end{trivlist}}
\newenvironment{problem}[2][Problema]{\begin{trivlist}
\item[\hskip \labelsep {\bfseries #1}\hskip \labelsep {\bfseries #2.}]}{\end{trivlist}}
\newenvironment{question}[2][Pregunta]{\begin{trivlist}
\item[\hskip \labelsep {\bfseries #1}\hskip \labelsep {\bfseries #2.}]}{\end{trivlist}}
\newenvironment{corollary}[2][Corolario]{\begin{trivlist}
\item[\hskip \labelsep {\bfseries #1}\hskip \labelsep {\bfseries #2.}]}{\end{trivlist}}
 
\begin{document}
 
% --------------------------------------------------------------
%                         Start here
% --------------------------------------------------------------
 
\title{Tarea Semanal 4}%replace X with the appropriate number
\author{Karel Zapfe\\ %replace with your name
Introducción al Caos y Dinámica No Lineal} %if necessary, replace with your course title
 
\maketitle

La \emph{Scholarpedia} es un intento de crear una \emph{wiki},
pero revisada por pares, de forma que la información que se encuentra ahí 
sea confiable y citable en artículos. Desgraciadamente las contribuciones
son muy escazas todavía. Vamos a darle un poco de público. \\
(\url{http://www.scholarpedia.org})

\begin{problem}{Lecturas}

Leer los siguientes artículos de la \emph{Scholarpedia}:

\begin{itemize}
\item Transitividad Topológica \\
  \url{http://www.scholarpedia.org/article/Topological_transitivity}
\item Dinámica simbólica\\
  \url{http://www.scholarpedia.org/article/Symbolic_dynamics}
  \item Dinámica hiperbólica \\
    \url{http://www.scholarpedia.org/article/Hyperbolic_dynamics}
\end{itemize}
Los artículos están escritos en un lenguaje sofisticado y general, y son
pesados de leer. Sin embargo es muy importante que afiancen estos conceptos,
asi que hagan un esfuerzo y hagan diagramas de los razonamientos dificiles.
\end{problem}

\begin{question}{1}

La medida que preserva el mapeo de Bernoulli es un poco
latosa de deducir, pero el mapeo es continuo, asi que podemos
suponer que la medida está definida sobre los Borelianos\ldots
\begin{itemize}
  \item ¡¿Qué qué?! Parafrasear lo que escribí arriba en un lenguaje
    menos formal. Recuerda, ¿Qué quiere decir continuidad? 
    ¿Cuales son los abiertos que nos gustan a los físicos?
    Demuestra entonces que se cumple la nocion de continuidad usando
    esa definición (algo sobre ``preservar''\ldots).
  \item El mapeo es topológicamente transitivo. Demostrar.
  \item A nosotros nos gustan los dibujos. Ilustra la demostración anterior.

\end{itemize}      

\end{question}


\end{document}