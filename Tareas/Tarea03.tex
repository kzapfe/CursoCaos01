% --------------------------------------------------------------
% This is all preamble stuff that you don't have to worry about.
% Head down to where it says "Start here"
% --------------------------------------------------------------
 
\documentclass[12pt]{article}
 
\usepackage[margin=1in]{geometry} 
\usepackage{amsmath,amsthm,amssymb}
\usepackage[utf8]{inputenc}
\usepackage[spanish]{babel}
\usepackage{graphicx} 
\usepackage{float} %Forzamos a las figuras a estar DONDE QUEREMOS

\newcommand{\N}{\mathbb{N}}
\newcommand{\Z}{\mathbb{Z}}
\newcommand{\M}{\mathbb{M}}
\newcommand{\rd}{\!\!\mathrm{d}}
 
\newenvironment{theorem}[2][Teorema]{\begin{trivlist}
\item[\hskip \labelsep {\bfseries #1}\hskip \labelsep {\bfseries #2.}]}{\end{trivlist}}
\newenvironment{lemma}[2][Lema]{\begin{trivlist}
\item[\hskip \labelsep {\bfseries #1}\hskip \labelsep {\bfseries #2.}]}{\end{trivlist}}
\newenvironment{exercise}[2][Ejercicio]{\begin{trivlist}
\item[\hskip \labelsep {\bfseries #1}\hskip \labelsep {\bfseries #2.}]}{\end{trivlist}}
\newenvironment{problem}[2][Problema]{\begin{trivlist}
\item[\hskip \labelsep {\bfseries #1}\hskip \labelsep {\bfseries #2.}]}{\end{trivlist}}
\newenvironment{question}[2][Pregunta]{\begin{trivlist}
\item[\hskip \labelsep {\bfseries #1}\hskip \labelsep {\bfseries #2.}]}{\end{trivlist}}
\newenvironment{corollary}[2][Corolario]{\begin{trivlist}
\item[\hskip \labelsep {\bfseries #1}\hskip \labelsep {\bfseries #2.}]}{\end{trivlist}}
 
\begin{document}
 
% --------------------------------------------------------------
%                         Start here
% --------------------------------------------------------------
 
\title{Tarea Semanal 3}%replace X with the appropriate number
\author{Karel Zapfe\\ %replace with your name
Introducción al Caos y Dinámica No Lineal} %if necessary, replace with your course title
 
\maketitle

Desempolvemos y mejoremos la simulación del péndulo que hicimos para la Tarea 1.
Si alguno de ustedes no la consiguió hacer bien, este es el momento para
formalizarla y dejarla perfecta. Si queremos tener una simulación
super precisa, le pedimos a Alejandra que nos comparta  su ejemplo de como lo resolvió usando
Runge-Kutta. 

El espacio fase del péndulo es un cilindoro, ya que la ``posición'' es una variable angular,
pero el momento es una variable potencialmente no acotada, es decir, un número real. 
Esto quiere decir que hay que identificar extremos opuestos del eje $q=:\theta$. 
En particular a mi me gusta identificar $\theta=\-pi$ con $\theta=pi$, de forma que el punto
de equilibrio sea $\theta=0$  y el punto inestable el ángulo opuesto, es decir,
el ángulo de identificación. Cada uno es libre de usar su propia convención, pero
de esta forma las ecuaciones Lagrangianas quedan con la forma particularmente simple 
del primer dia. En mi notación $l$ es el momento (angular) conjugado a $\theta$. Sea
$L$ la longitud del brazo del péndulo y $m$ su masa:
\begin{equation}
H(\theta, l)= \frac{l^2}{2m R^2}-mgR \cos\theta
\end{equation}
Podemos escoger unidades tales que $mgR=1$ y $mR^2=1$. Con eso el sistema termina con 
la siguientes ecuaciones de movimiento:
\begin{align}
\dot{l} & = -\frac{\partial H}{\partial \theta}  =  -\sin \theta, \\
\dot{\theta} & = \frac{\partial H}{\partial l} = l. \\
\end{align}

Dado que el sistema es conservativo, básicamente la energia indexa las trayectorias.

\begin{figure}[H]
\begin{center}
\includegraphics[width=0.9\textwidth]{EspacioPenduloSimple01.png}
\end{center}
\caption{ El espacio fase del Péndulo, fibrado por variedades de energia constante.
En este caso, coinciden con las trayectorias}.
\end{figure}

Les deje en \verb:github: dos archivos auxiliares, aparte de la fuente de este mismo
archivo, para que se inspiren. El pendulo simple, resuelto a lo Euler en python,
en \verb:PenduloSimple.py:, y como hice el dibujo, en \verb:GraficandoPenduloSimple01.gpl:.


\begin{question}{1}
Veamos que tan mezclante es el sistema, si es que lo es. 
Considera un pequeño rectangulo de condiciones inciciales, centrado
en algun lugar del espacio fase. 
\begin{itemize} 
\item Coloca tu cuadrado de condiciones iniciales en un lugar del espacio fase,
y muestrame sus imagenes para varias selecciones de $t$. Deja algunas
lineas de energia constante debajo, como referencia visual,
para que veamos por donde se mueve el cuadrado. Prueba para $t$ pequeño (del
orden de tu $dt$),
$t$ mediano (del orden de 10 o 100 $dt$s), y muy grande. ¿Qué observas?
\item Repite lo anterior, pero busca lugares especiales del espacio fase
para colocarlo. Ya sabes cuales. ¿Cómo le llamarías a las figuras producidas?
\item El sistema, sobre el espacio fase total, no puede ser mezclante, porque es
conservativo. Explica este punto.
\item ¿Podría ser mezclante en cada ``superficie'' de energía constante?
\end{itemize}
\end{question}
 
\begin{question}{2}
Dibujen el diagrama de las escaleras y caracoles para el mapeo de Bernoulli.
\begin{itemize}
\item Si exageran en el número de lineas, queda muy claro algo... ¿qué?
\item Busquen una orbita periodica de periodo medio, por decir algo, 10. Hagan el diagrama.
\item ¿Que pasa si prueban hacer algo similar a lo del problema anterior? Usen una linea
de condiciones iniciales. El sistema es mezclante, aunque la medida conservada no sea intuitiva. 
De todas maneras, algo se debe de poder percibir. Inventen un análisis gráfico que
delate esta propiedad.
\end{itemize}  
\end{question}

\begin{question}{3}
Hay un mapeo que es algo así como una mezcla del mapeo de Bernoulli y el mapeo
logístico con $k=4$. Se llama el mapeo de la tienda y esta dado por la regla siguiente,
en el intervalo unitario.
\begin{equation}
T(x) = \begin{cases} 2x &\mbox{si } x < 1/2 \\ 
2(1-x) & \mbox{si } 1/2 <x \leq 1. \end{cases} 
\end{equation}
\begin{itemize}
\item ¿Porqué se llama el mapeo de la tienda?
\item ¿Porque digo que es algo así como una mezcla de los otros dos?
\item ¿Qué tan lejos está del mapeo de Bernoulli? ¿Lo puedes transformar en aquél?
(Esto sería, en lenguaje matematico moderno, buscar una biyección entre los dos sistemas
dinámicos tal que sean uno conjugado del otro).
\item  Punto Extra: Si se parece al mapeo logístico, tal vez podemos reducir el tamaño de la tienda
con algun parámetro... y hacer un diagrama de bifurcaciones. 
\end{itemize}

\end{question}

\end{document}