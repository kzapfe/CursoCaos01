\documentclass[letterpaper,10pt, jcp, aps, preprint]{revtex4-1}

\usepackage[utf8]{inputenc}
\usepackage[spanish]{babel}
\usepackage{amsmath, amsfonts}
\usepackage{graphicx}
\usepackage{natbib}
\usepackage{caption}
\usepackage{subcaption}
%\usepackage{authblk} parece ser que ya esta metido cuando usas preprint



%\bibliographystyle{alpha}

%opening

%\author{Rosa Rodríguez \& David P.~Sanders \& W. P. K. Zapfe}
%\affil{Departamento de Física, Facultad de Ciencias, Universidad Nacional Autónoma de México, Ciudad Universitaria, Del.~Coyoacán, México D.F. 04510, Mexico}

\usepackage{mathptmx}
% I dont know if that package is compatible with revtex.

\newcommand{\defeq}{:=}
\newcommand{\mean}[1]{\left \langle #1 \right \rangle}
\newcommand{\rd}{\!\!\!\!\mathrm{d}}
\newcommand{\RR}{\mathbb{R}}
\newcommand{\vv}{\mathbf{v}}
\newcommand{\indicator}[1]{\mathbf{1}_{ \{   #1 \} } } 
\newcommand{\etal}{\emph{et al.\ }} 

\setlength{\parskip}{10pt}
\setlength{\parindent}{0pt}



\begin{document}


\title{Tarea Uno}
\author{Karel Zapfe}
\email{karelz@fis.unam.mx}


\begin{abstract}
A entregar el próximo miércoles, 20 de agosto. Se vale entregar a mano, o por
correo electrónico, antes de las 16 horas. Formatos aceptables, pdf o txt (texto crudo)
para el texto, cualquier formato de imagenes no muy pesado (incluyendo dibujos a mano)
para las graficas. \\
Voy a ir poniendo programas ejemplo y ayudas computacionales en github:\\
\url{https://github.com/kzapfe//CursoCaos01}.\\
Sed libres de usarlos, comentarlos e incluso mejorarlos.
\end{abstract}

\maketitle

\section{Ejercicio:}



La ecuación de un péndulo simple en unidades ``naturales'' es
\begin{equation}
\ddot{\theta}+\omega\sin\theta=0.
\end{equation}
Usando una variable auxiliar $p\defeq \dot{\theta}$ podemos escribir
esto como un par de ecuaciones parciales ordinarias de primer orden:
\begin{align}
\dot{\theta} &=p\\
\dot{p} &= -\omega \sin\theta.
\end{align}
Por cierto, este truco se puede usar siempre. Una ecuación diferencial
de orden $n$ es equivalente, de esta forma, a $n$ ecuaciones 
diferenciales de orden $1$.

\subsection{}
Haz una gráfica de las órbitas de las soluciones en el espacio
$(\theta, p)$, para varias condiciones iniciales. Indica las condiciones
iniciales.

\subsection{}
¿Recuerdas la aproximación para $\theta\ll 1$? Has una gráfica también en el
mismo espacio de ésta. Compara las curvas de ambos espacios (si quieres, encimalos). 
¿Qué notas?

\subsection{}
Hay dos puntos muy especiales en la grafica del problema ``A''. ¿Cuáles?
¿Qué representan? ¿Cómo los llamarías?

\section{Ejercicio:}

Leete el texto adjunto. Enjoy. 

\end{document}